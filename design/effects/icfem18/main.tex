\documentclass[runningheads]{llncs}

\usepackage{graphicx}
\usepackage{times}
\usepackage{bm}
\usepackage{color}
\usepackage{ebproof} % For proof trees
\usepackage{listings} % For code snippets
\usepackage{proof} % For inference rules.
\usepackage[ruled]{algorithm2e}
\usepackage{amssymb}
\usepackage{amsmath}

\definecolor{grey}{gray}{0.92}

\lstset{
tabsize=3,
basicstyle=\ttfamily\small, commentstyle=\itshape\rmfamily, 
backgroundcolor=\color{grey},
numbers=left,
numberstyle=\tiny,
language=java,
moredelim=[il][\sffamily]{?},
mathescape=true,
showspaces=false,
showstringspaces=false,
columns=fullflexible,
escapeinside={(@}{@)}, morekeywords=[1]{def, if, then, else, with, val, module, instantiate, require, let, in}}
\lstloadlanguages{Java,VBScript,XML,HTML}

% omitting text
\newcommand{\OMIT}[1]{}

%using \kwa outside math mode
\newcommand{\kwat}[1]{$\kwa{#1}$}

% Hyphens
\newcommand{\hyphen}{\hbox{-}}

% For defining derived forms.
\newcommand\defn{\mathrel{\overset{\makebox[0pt]{\mbox{\normalfont\tiny\sffamily def}}}{=}}}

% Constants, types.
\newcommand{\unit}{\kwa{unit}}
\newcommand{\Unit}{\kwa{Unit}}
\newcommand{\File}{\kwa{File}}
\newcommand{\Socket}{\kwa{Socket}}

% Keywords.
\newcommand{\kwa}[1]{\mathtt{#1}}
\newcommand{\kw}[1]{\mathtt{#1}~}

% Expressions.
\newcommand{\import}[4]{\kwa{import}(#1)~#2 = #3~\kw{in} #4}
\newcommand{\letxpr}[3]{\kw{let} #1 = #2~\kw{in} #3}	

% Functions in the type theory.
\newcommand{\annot}[2]{\kwa{annot}(#1, #2)}
\newcommand{\erase}[1]{\kwa{erase}(#1)}
\newcommand{\fx}[1]{\kwa{effects}(#1)}
\newcommand{\hofx}[1]{\kwa{ho \hyphen effects}(#1)}

% Safety predicates in the type theory.
\newcommand{\safe}[2]{\kwa{safe}(#1, #2)}
\newcommand{\hosafe}[2]{\kwa{ho \hyphen safe}(#1, #2)}

% Names of the calculi.
\newcommand{\opercalc}{\kwa{OC}}
\newcommand{\epscalc}{\kwa{CC}}

\renewcommand{\algorithmcfname}{ALGORITHM}
\SetAlFnt{\small}
\SetAlCapFnt{\small}
\SetAlCapNameFnt{\small}
\SetAlCapHSkip{0pt}
\IncMargin{-\parindent}

%% Bibliography style
\bibliographystyle{splncs04}


\begin{document}

\title{Capabilities: Effects for Free}

\author{Aaron Craig\inst{1} \and
  Alex Potanin\inst{1}\orcidID{0000-0002-4242-2725} \and
  Lindsay Groves\inst{1} \and
  Jonathan Aldrich\inst{2}\orcidID{0000-0003-0631-5591}}

\authorrunning{A. Craig et al.}

\institute{School of Engineering and Computer Science, Victoria University of Wellington, NZ\\
  \email{\{aaron.craig,alex,lindsay\}@ecs.vuw.ac.nz} \and
  School of Computer Science, Carnegie Mellon University\\
  \email{jonathan.aldrich@cs.cmu.edu}}

\maketitle

\begin{abstract}
  Object capabilities are increasingly used to reason informally about
  the properties of secure systems. Bur can capabilities also aid in
  \textit{formal} reasoning? To answer this question, we examine a
  calculus that uses effects to capture resource use and extend it to
  support capability-based
  reasoning. We demonstrate that capabilities provide a way to reason
  about effects: we can bound the effects of an expression
  based on the capabilities to which it has access.  This reasoning is
  ``free'' in that it relies only on type-checking (not
  effect-checking), does not require the programmer to add effect
  annotations within the expression, and does not require the
  expression to be analysed for its effects. Our result sheds light on
  the essence of what capabilities provide and suggests ways of
  integrating lightweight capability-based reasoning into languages.

% \keywords{capabilities, effects, type systems}
\end{abstract}

\section{Introduction}

Capabilities have been recently gaining new attention as a promising
mechanism for controlling access to resources, particularly in
object-oriented languages and
systems~\cite{miller03,drossopoulou07,dimoulas14,devriese16}.  A
\textit{capability} is an unforgeable token that can be used by its
bearer to perform some operation on a resource \cite{dennis66}.  In a
\textit{capability-safe} language, all resources must be accessed
through object capabilities, and a resource-access capability must be
obtained from an object that already has it: ``only connectivity
begets connectivity'' \cite{miller03}.  For example, a \kwat{logger}
component that provides a logging service would need to be initialised
with an object capability providing the ability to append to the log
file.

Capability-safe languages prohibit the \textit{ambient
  authority}~\cite{miller06} that is present in non-capability-safe
languages. An implementation of a \kwat{logger} in Java, for example,
does not need to be initialised with a log file capability, as it can simply
import the appropriate file-access library and open the log file for
appending by itself. But critically, a malicious implementation could also
delete the log, read from another file, or exfiltrate logging information
over the network.  Other mechanisms such as sandboxing can be used
to limit the damage of such malicious components, but recent work has
found that Java's sandbox (for instance) is difficult to use and
therefore often misused~\cite{coker15,maass16}.

In practice, reasoning about resource use in capability-based systems
is mostly done informally.
%% TODO: Cite with some evidence? Lindsay says.
But if capabilities are useful for \textit{informal} reasoning,
shouldn't they also aid in \textit{formal} reasoning?  Recent work by
Drossopoulou et. al.~\cite{drossopoulou07} sheds some light on this question
by presenting a logic that 
formalizes capability-based reasoning about trust between
objects.  Two other trains of work, rather than
formalise capability-based reasoning itself, reason about how
capabilities may be used: Dimoulas et al.~\cite{dimoulas14} developed a formalism for
reasoning about which components may use a capability and which may
influence (perhaps indirectly) the use of a
capability, while  Devriese et al.~\cite{devriese16} formulate an effect
parametricity theorem that limits the effects of an object based on
the capabilities it possesses, and then use logical relations to
reason about capability use in higher-order settings.  Overall, this prior work presents new formal systems for reasoning
about capability use, or reasoning about new properties using
capabilities.

We are interested in a different question: can capabilities be used to
enhance formal reasoning that is currently done without relying on
capabilities?  In other words, what value do capabilities add to
existing formal reasoning approaches?

To answer this question, we
decided to pick a simple and practical formal reasoning system, and 
see if capability-based reasoning could help.  A natural choice for our
investigation is effect systems~\cite{nielson99}.  Effect systems
are a relatively simple formal reasoning approach, and keeping things
simple will help to highlight the difference made by capabilities.
 Effects also have an intuitive link to capabilities: in a system that
 uses capabilities to protect resources, an expression can only have
 an effect on a resource if it is given a capability to do so.

One challenge to the wider adoption of effect systems is their annotation
overhead~\cite{rytz12}. For example, Java's checked exception system, which
is a kind of effect system, is often criticised for being cumbersome~\cite{Kiniry2006}. While effect inference can be used to reduce the annotations required~\cite{koka14},
understanding error messages that arise through effect inference requires a detailed understanding of the internal structure of the code, not just its interface. Capabilities are a promising
alternative for reducing the overhead of effect annotations, as
suggested by the following example:

\begin{lstlisting}
import log : String -> Unit with effect File.write
e
\end{lstlisting}

This code, like the rest in this paper, is written in a capability-safe language supporting
first-class, object-like modules, similar to \textit{Wyvern}~\cite{kurilova16}. In it, the
expression \kwat{e} declares what capabilities it needs to execute. In this case,
\kwat{e} must be passed a function of type \kwat{String \rightarrow Unit}, which
incurs no more than the single \kwat{File.write} effect when invoked. This function is
bound to the name \kwat{log} inside \kwat{e}.

If we were to evaluate \kwat{e}, what could we say about its effects on resources,
such as the file system or network? Because we are in a capability-safe language,
\kwat{e} has no ambient authority, so the only way it could have any effects is via
the \kwat{log} function given to it. Since the \kwat{log} function is annotated as
having no more than the \kwat{File.write} effect, this is an upper-bound on the
effects of \kwat{e}. Note we only required that \kwat{e} obeys the rules of
capability safety. We did not require it to have effect annotations, and we
didn't analyse its structure, as an effect inference would. Also note that \kwat{e}
might be arbitrarily large, perhaps consisting of an entire program we have
downloaded from a source we trust enough to write to a log, but not enough to
access any other resources. Thus in this scenario, capabilities can be used to
reason ``for free'' about the effects of a large body of code (\kwat{e}), based on a
few annotations on the components it imports (\kwat{log}).

This example illustrates the central intuition of this paper: in a capability-safe setting,
the effects of an unannotated expression can be bounded by the effects latent in the
variables that are in scope. In the remainder of this paper, we formalise these ideas
in a capability calculus (\kwat{CC}; section 2) and prove it sound. Along the way we
must generalise this intuition: what if \kwat{log} takes a higher-order argument? If
\kwat{e} evaluates, not to \kwat{unit}, but to a function, what can we say about its
effects?

While the current resurgence of interest in capabilities is primarily focused on
object-oriented languages, for simplicity our formal definitions build on a typed
lambda calculus with a simple notion of capabilities and their operations. \kwat{CC}
permits the nesting of unannotated code inside annotated code in a controlled,
capability-safe manner using the \kwat{import} form from above. This allows us to
reason about unannotated code by inspecting what capabilities are passed into it
from its unannotated surroundings. We then show how \kwat{CC} can model
practical situations by encoding a range of Wyvern-like programs. A more thorough
discussion of the proofs and encodings are given in an accompanying technical
report~\cite{ecs:2018:aaron-tr}.

\section{Capability Calculus ($\epscalc$)}
\label{s:cc}

While the current resurgence of interest in capabilities is primarily focused on
object-oriented languages, for simplicity our formal definitions build on a typed
lambda calculus with a simple notion of capabilities and their operations. \kwat{CC}
permits the nesting of unannotated code inside annotated code in a controlled,
capability-safe manner using the \kwat{import} form from Figure
\ref{fig:declaring_effects}. This allows us to reason about unannotated code
by inspecting what capabilities are passed into it from its unannotated
surroundings.

Allowing effect-annotated and unannotated code to be mixed helps
reduce the cognitive overhead on developers, allowing them to 
prototype in the unannotated sublanguage and incrementally add
annotations as they are needed. Reasoning about unannotated code is
difficult in general. Figure \ref{fig:unannotated_reasoning}
demonstrates why: $\kwa{apply}$ takes a function $f$ as input and
executes it, but the effects of $f$ depend on its
implementation. Without more information, there is no way to know what
effects might be incurred by $\kwa{apply}$.

\begin{figure}
\vspace*{-5mm}
\begin{lstlisting}
def apply(f: Unit $\rightarrow$ Unit):
   f()
\end{lstlisting}
\vspace*{-5mm}
\caption{What effects can $\kwa{apply}$ incur?}
\vspace*{-5mm}
\label{fig:unannotated_reasoning}
\end{figure}

Consider another scenario, where a developer must decide whether or not
to use the \kwat{logger} functor defined in Figure \ref{fig:cc_motivation}. This
functor takes two capabilities as input, \kwat{File} and \kwat{Socket}.\footnote{Note that the resource literal is \kwat{File}, while the type of the resource literal is
\kwat{\{File\}}.} It instantiates an object-like module that has a single, unannotated \kwat{log} method with access to these capabilities. The type of this object-like module is \kwat{Logger}, which is assumed to be defined elsewhere.

\begin{figure}
\vspace*{-5mm}
\begin{lstlisting}
module def logger(f:{File},s:{Socket}):Logger
def log(x: Unit): Unit
   ...
\end{lstlisting}
\vspace*{-5mm}
\caption{In a capability-safe setting, $\kwa{logger}$ can only exercise authority over the $\kwa{File}$ and $\kwa{Socket}$ capabilities given to it.}
%\vspace*{-5mm}
\label{fig:cc_motivation}
\end{figure}

How can we determine what effects will be incurred if $\kwa{Logger.log}$ is
invoked? One approach is to manually\footnote{or automatically---but if the
  automation produces an unexpected result we must fall back to manual
  reasoning to understand why.} examine its source code, but this is
tedious and error-prone. In many real-world situations, the source
code may be obfuscated or unavailable. A capability-based argument can do
better, since a \kwat{Logger} can only exercise the authority it is explicitly
given. In this case, the \kwat{logger} functor must be given \kwat{File} and
\kwat{Socket}, so an upper bound on the effects of the \kwat{Logger} it
instantiates will be the set of all operations on those resources,
\kwat{\{File.*, Socket.*\}}. Knowing the \kwat{Logger} could perform
arbitrary reads and writes to \kwat{File}, or communicate with \kwat{Socket}, the
developer decides this implementation cannot be trusted and does not use it.

To model this situation in
$\epscalc$, we add a new $\kwa{import}$ expression
that selects what authority $\varepsilon_s$ the unannotated code may
exercise. In the above example, the expected least authority of
$\kwa{Logger}$ is $\{ \kwa{File.append} \}$, so that is what the
corresponding $\kwa{import}$ would select. The type system can then
check whether the capabilities being passed into the unannotated code
exceed $\varepsilon_s$. If it accepts, then $\varepsilon_s$ is a safe
upper bound on the effects of the unannotated code. This is the
key result: when unannotated code is nested inside annotated code,
capability-safety enables us to make a safe inference about its
effects by examining what capabilities are being passed in by the
annotated code.

\subsection{Grammar ($\epscalc$)}

The grammar of $\epscalc$ has rules for annotated code and
analogous rules for unannotated code. To distinguish the two, we put a
hat above annotated types, expressions, and contexts. $\hat e$,
$\hat \tau$, and $\hat \Gamma$ are annotated, while $e$, $\tau$, and
$\Gamma$ are unannotated. The rules for unannotated programs and their
types are given in Figure
\ref{fig:epscalc_unannotated_grammar}. Unannotated types $\tau$ are
built using $\rightarrow$ and sets of resources $\{ \bar r \}$. An
unannotated context $\Gamma$ maps variables to unannotated types.
The syntax for invoking an operation on a resource is $e.\pi$. Resource
literals and operations are drawn from fixed sets $R$ (containing, e.g.
\kwat{File}, \kwat{Socket}) and $\Pi$ (containing, e.g. \kwat{write},
\kwat{read}).

\begin{figure}[htb]
\vspace*{-5mm}
\[
\begin{array}{lll}
\begin{array}{lllr}
e & ::= & ~ & exprs: \\
	& | & x & variable \\
	& | & v & value \\
	& | & e ~ e & application \\
	& | & e.\pi & operation \\
	&&\\

v & ::= & ~ & values: \\
	& | & r & resource~literal \\
	& | & \lambda x: \tau.e & abstraction \\
&&\\
\end{array}
\hspace{5ex}
\begin{array}{lllr}
\tau & ::= & ~ & types: \\
		& | & \{ \bar r \} & resource~set \\
		& | & \tau \rightarrow \tau & function\\ 
		&&\\

\Gamma & ::= & ~ & type~ctx: \\
				& | & \varnothing & empty~ctx\\
				& | & \Gamma, x: \tau & binding\\
				&&\\
				
\varepsilon & ::= & ~ & effects: \\
		& | & \{ \overline{r.\pi} \} & effect~set
\end{array}
\end{array}
\]
\vspace*{-5mm}
\caption{Unannotated programs and types in $\epscalc$.}
\vspace*{-5mm}
\label{fig:epscalc_unannotated_grammar}
\end{figure}

Because our focus is on tracking what effects happen, i.e. whether
particular operations are invoked on particular resources, we make the
following simplifying assumptions: first, any operation may be called on any
resource literal; and second, all operations take no inputs and return \kwat{unit}.

\begin{figure}[hbt]
\vspace*{-5mm}
\[
\begin{array}{lll}
\begin{array}{lllr}
\hat e & ::= & ~ & labelled~exprs: \\
	& | & x & variable \\
	& | & \hat v & value \\
	& | & \hat e ~ \hat e & application \\
	& | & \hat e.\pi & operation \\
	& | & \kwa{import}(\varepsilon_s)~x = \hat e~\kwa{in}~e & import \\
	&&\\

\hat v & ::= & ~ & labelled~values: \\
	& | & r & resource~literal\\
	& | & \lambda x: \hat \tau.\hat e & abstraction \\
\end{array}
& ~~~~~~~~&
\begin{array}{lllr}

\hat \tau & ::= & ~ & annotated ~types: \\
		& | & \{ \bar r \} & resource~set\\
		& | & \hat \tau \rightarrow_{\varepsilon} \hat \tau & function \\
		&&\\

\hat \Gamma & ::= & ~ & annotated~type~ctx:\\
				& | & \varnothing & empty~ctx.\\
				& | & \hat \Gamma, x: \hat \tau & binding \\
				&&\\

\varepsilon & ::= & ~ & effects: \\
		& | & \{ \overline{r.\pi} \} & effect~set \\
\end{array}
\end{array}
\]
\vspace*{-5mm}
\caption{Annotated programs and types in $\epscalc$.}
\vspace*{-5mm}
\label{fig:epscalc_annotated_grammar}
\end{figure}

Rules for annotated programs and their types are shown in
Figure \ref{fig:epscalc_annotated_grammar}. The first main difference is that
the $\rightarrow_{\varepsilon}$ type constructor has a subscript
$\varepsilon$, which is a set of effects that functions of that type may incur.
The other main difference is the new expression form, 
$\import{\varepsilon_s}{x}{\hat e}{e}$, where $e$ is some unannotated code and
$\hat e$ is a capability being passed to it; we call $\hat e$ an import. For
simplicity, we assume there is only ever one import. Note the definition
not only allows resource literals to be imported, but also effectful functions.
Inside $e$, $\hat e$ is bound to the variable $x$. $\varepsilon_s$
is the maximum authority that $e$ is allowed to exercise (its ``selected authority'').
For example, suppose an
unannotated $\kwa{Logger}$, which requires $\kwa{File}$, is expected
to only $\kwa{append}$ to a file, but has an implementation which
writes. This would be the expression
$\import{\kwa{File.append}}{x}{\kwa{File}}{\lambda y:
  \Unit.~\kwa{x.write}}$. The $\kwa{import}$ expression is the only way to
mix annotated and unannotated code, because it is the only situation in which
we can say something interesting about the effects of unannotated code. For the
rest of our discussion of $\epscalc$, we will only be interested in unannotated
code when it is encapsulated by an $\kwa{import}$ expression.

Capability safety prohibits ambient authority. $\epscalc$
meets this requirement by forbidding the use of resource literals directly inside
an \kwat{import} expression (though they can still be passed in as a capability
via the binding variable $x$). We could have enforced this syntactically, but
we choose to do it using the typing rule for \kwat{import} in section 2.3.

\subsection{Semantics ($\epscalc$)}

The rules for $\epscalc$ are natural extensions of the simply-typed lambda calculus,
so for brevity we only give the rules for \kwat{import} (see Figure
\ref{fig:epscalc_reductions}). Reductions are defined 
on annotated expressions, using the notation $\hat e \longrightarrow \hat e'~|~
\varepsilon'$, which means that $\hat e$ is reduced to $\hat e'$ in a single
step, incurring the set of effects $\varepsilon'$. To execute the unannotated code
inside an \kwat{import} expression, we recursively annotate its components with the
selected authority $\varepsilon_s$. While it is meaningful to execute
unannotated code, we only care about it inside
\kwat{import} expressions, so do not bother to give rules for this.

\textsc{E-Import1} reduces the capability being imported. When it has been
reduced to a value $\hat v$, \textsc{E-Import2} annotates $e$ with the selected
authority $\varepsilon$ --- this is $\annot{e}{\varepsilon}$ --- and
substitutes the import $\hat v$ for its name $x$ in $e$ --- this is
$[\hat v/x]\annot{e}{\varepsilon}$.

\begin{figure}
\vspace*{-5mm}
\fbox{$\hat e \longrightarrow \hat e~|~\varepsilon$}
\[
\begin{array}{c}
\infer[\textsc{(E-Import1)}]
	{\kwa{import}(\varepsilon_s)~x = \hat e~\kw{in} e \longrightarrow \kwa{import}(\varepsilon_s)~x = \hat e'~\kw{in} e~|~\varepsilon'}
	{\hat e \longrightarrow \hat e'~|~\varepsilon'}\\[2ex]

\infer[\textsc{(E-Import2)}]
	{\kwa{import}(\varepsilon_s)~x = \hat v~\kw{in} e \longrightarrow [\hat v/x]\kwa{annot}(e, \varepsilon_s)~|~\varnothing}
	{}
\end{array}
\]
\vspace*{-5mm}
\caption{New single-step reductions in $\epscalc$.}
\vspace*{-5mm}
\label{fig:epscalc_reductions}
\end{figure}

$\annot{e}{\varepsilon}$ is the expression obtained by
recursively annotating the parts of $e$ with the set of effects
$\varepsilon$. A definition is given in Figure
\ref{fig:annot_defn}, with versions defined on expressions and types.
 Later we will need to annotate contexts, so
the definition is given here. Note that
$\kwa{annot}$ operates on a purely syntatic level. Nothing prevents
us from annotating a program with something unsafe, so any use of
$\kwa{annot}$ must be justified.

\begin{figure}
\vspace*{-5mm}
\begin{itemize}
	\setlength\itemsep{-0.2em}
\item[] $\kwa{annot} :: e \times \varepsilon \rightarrow \hat e$
	\item[] $\annot{r}{\_} = r$
	\item[] $\annot{\lambda x: \tau_1 . e}{\varepsilon} = \lambda x: \annot{\tau_1}{\varepsilon} . \annot{e}{\varepsilon}$
	\item[] $\annot{e_1~e_2}{\varepsilon} = \kwa{annot}(e_1, \varepsilon)~\kwa{annot}(e_2, \varepsilon)$
	\item[] $\annot{e_1.\pi}{\varepsilon} = \annot{e_1}{\varepsilon}.\pi$
\end{itemize}
	
\vspace{-5mm}

\begin{itemize}
	\setlength\itemsep{-0.2em}
\item[] $\kwa{annot} :: \tau \times \varepsilon \rightarrow \hat \tau$
	\item[] $\annot{\{ \bar r \}}{\_} = \{ \bar r \}$
	\item[] $\annot{\tau_1 \rightarrow \tau_2}{\varepsilon} = \annot{\tau_1}{\varepsilon} \rightarrow_{\varepsilon} \annot{\tau_2}{\varepsilon}$.	
\end{itemize}

\vspace{-5mm}

\begin{itemize}
	\setlength\itemsep{-0.2em}
\item[] $\kwa{annot} :: \Gamma \times \varepsilon \rightarrow \hat \Gamma$
	\item[] $\annot{\varnothing}{\_} = \varnothing$
	\item[] $\annot{\Gamma, x: \tau}{\varepsilon} = \annot{\Gamma}{\varepsilon}, x: \annot{\tau}{\varepsilon}$
\end{itemize}
\vspace*{-5mm}
\caption{Definition of $\kwa{annot}$.}
\vspace*{-5mm}
\label{fig:annot_defn}
\end{figure}

\subsection{Static Rules ($\epscalc$)}

Terms can be annotated or unannotated, so we need to be able to
recognise when either is well-typed. We do not reason about the
effects of unannotated code directly, so judgements involving them
only ascribe a type to an expression, with the form $\Gamma \vdash e: \tau$.
Subtyping judgements have the form $\tau <: \tau$. Because these rules are
essentially those of the simply-typed lambda calculus, we do not list
them here.

Judgements involving annotated terms have the form
$\hat \Gamma \vdash \hat e: \hat \tau~\kw{with} \varepsilon$, meaning that
when $\hat e$ is evaluated, it reduces to a value of type $\hat \tau$,
incurring no more than the effects in $\varepsilon$. 
Most of the rules are analogous to those of the simply-typed lambda calculus;
these ones are given in Figure \ref{fig:cc_static_rules}. Note that the rule for
typing an operation call, \textsc{$\varepsilon$-OperCall}, types the expression
as \kwat{Unit}, following our simplifying assumption that all operations return
\kwat{Unit}.

\begin{figure}
\vspace*{-5mm}
\noindent
\fbox{$\Gamma \vdash e: \tau~\kw{with} \varepsilon$}
\vspace{-0mm}
\[
\begin{array}{c}
\infer[\textsc{($\varepsilon$-Var)}]
	{ \Gamma, x:\tau \vdash x: \tau~\kw{with} \varnothing }
	{}
	
	~~~~~~~
	
\infer[\textsc{($\varepsilon$-Resource)}]
 	{ \Gamma, r: \{ r \} \vdash r : \{ r \}~\kw{with} \varnothing }
 	{} \\[3ex]
 	
	\infer[\textsc{($\varepsilon$-Abs)}]
	{ \Gamma \vdash \lambda x:\tau_2 . e : \tau_2 \rightarrow_{\varepsilon_3} \tau_3~\kw{with} \varnothing }
	{ \Gamma, x: \tau_2 \vdash e: \tau_3~\kw{with} \varepsilon_3 }
	
	~~~~~~~
	
\infer[\textsc{($\varepsilon$-App)}]
	{ \Gamma \vdash e_1~e_2 : \tau_3~\kw{with} \varepsilon_1 \cup \varepsilon_2 \cup \varepsilon  }
	{ \begin{array}{c}
            \Gamma \vdash e_1: \tau_2 \rightarrow_{\varepsilon} \tau_3~\kw{with} \varepsilon_1 \\[.6mm]
            \Gamma \vdash e_2: \tau_2~\kw{with} \varepsilon_2
          \end{array} } \\[3ex]
	
\infer[\textsc{($\varepsilon$-OperCall)}]
	{ \Gamma \vdash e.\pi: \kw{Unit} \kw{with} \{ \bar r.\pi \} }
	{ \Gamma \vdash e: \{ \bar r \} } 

	~~~~~~~

\infer[\textsc{($\varepsilon$-Subsume)}]
	{ \Gamma \vdash e: \tau' ~\kw{with} \varepsilon'}
	{ \begin{array}{c} 
            \Gamma \vdash e: \tau ~\kw{with} \varepsilon \\[.6mm]
            \tau <: \tau' ~~~ \varepsilon \subseteq \varepsilon'
          \end{array}}\\[3ex]
\end{array}
\]

\fbox{$\Gamma \vdash e: \tau~\kw{with} \varepsilon$}

\[
\begin{array}{c}
\infer[\textsc{(S-Arrow)}]
	{ \tau_1 \rightarrow_{\varepsilon} \tau_2 <: \tau_1' \rightarrow_{\varepsilon'} \tau_2' }
	{ \tau_1' <: \tau_1 & \tau_2 <: \tau_2' & \varepsilon \subseteq \varepsilon' }
~~~~~~
\infer[\textsc{(S-Resource)}]
	{ \{ \bar r_1 \} <: \{ \bar r_2 \} }
	{ r \in r_1 \implies r \in r_2 }
\end{array}
\]
\vspace*{-5mm}
\caption{Type-and-effect and subtyping judgements in $\epscalc$.}
\vspace*{-5mm}
\label{fig:cc_static_rules}
\end{figure}

There is one rule left, for typing \kwat{import}. Since it is a complicated
rule, we will start with a simplified (but incorrect) version, and spend the
rest of the section building up to the final version.

To begin, typing $\import{\varepsilon_s}{x}{\hat e}{e}$ in a context
$\hat \Gamma$ requires us to know that $\hat e$ is
well-typed, so we add the premise
$\hat \Gamma \vdash \hat e: \hat \tau~\kw{with} \varepsilon_1$.
$e$ is only allowed to use what authority has been explicitly given to it
(i.e. the capability $\hat e$, bound to $x$). To ensure this, we require
that $e$ can be typechecked using only one binding, $x: \hat \tau$,
which binds $x$ to the type of the capability being imported.
Typing $e$ in this restricted environment means it cannot use any
other capabilities, thus prohibiting the exercise of ambient authority.

There is a problem though: $e$ is unannotated, while $\hat \tau$ is
annotated, and there is no rule for typechecking unannotated code in
an annotated context. To get around this, we define a function
\kwat{erase} in Figure \ref{fig:erase_defn}, which removes the
annotations from a type. We can then add
$x: \erase{\hat \tau} \vdash e: \tau$ as a premise.

\begin{figure}
\vspace*{-5mm}
\begin{itemize}
	\setlength\itemsep{-0.2em}
\item[] $\kwa{erase} :: \hat \tau \rightarrow \tau$
	\item[] $\erase{\{ \bar r \}} = \{ \bar r \}$
	\item[] $\erase{\hat \tau_1 \rightarrow_{\varepsilon} \hat \tau_2} = \erase{\hat \tau_1} \rightarrow \erase{\hat \tau_2}$
\end{itemize}
\vspace*{-5mm}
\caption{Definition of $\kwa{erase}$.}
\vspace*{-5mm}
\label{fig:erase_defn}
\end{figure}

The first version of \textsc{$\varepsilon$-Import} is given in Figure
\ref{fig:import_rule_1}. Since
$\import{\varepsilon_s}{x}{\hat v}{e}$ reduces to $[\hat
v/x]\annot{e}{\varepsilon_s}$ by \textsc{E-Import2}, its ascribed type is $\annot{\tau}{\varepsilon}$, which is the type of the unannotated
code $e$, annotated with its selected authority $\varepsilon_s$. The
effects of reducing the $\kwa{import}$ are $\varepsilon_1 \cup \varepsilon_s$
--- the former happens when the imported capability is reduced to a value,
while the latter happens when the body of the \kwat{import} expression is
annotated and executed.

\begin{figure}[h]
\vspace*{-5mm}
\[
\begin{array}{c}
\infer[\textsc{($\varepsilon$-Import1-Bad)}]
	{ \hat \Gamma \vdash \import{\varepsilon_s}{x}{\hat e}{e}: \kwa{annot}(\tau, \varepsilon_s)~\kw{with} \varepsilon_s \cup \varepsilon_1 }
	{ \hat \Gamma \vdash \hat e: \hat \tau~\kw{with} \varepsilon_1 & x: \kwa{erase}(\hat \tau) \vdash e: \tau }

\end{array}
\]
\vspace*{-5mm}
\caption{A first (incorrect) rule for type-and-effect checking $\kwa{import}$ expressions.}
\vspace*{-5mm}
\label{fig:import_rule_1}
\end{figure}

This first rule is incomplete, since any capability can be passed to the unannotated
code $e$, even if it has effects that weren't declared in $\varepsilon_s$. To avoid
this, we define a function \kwat{effects}, which collects the
set of effects that an (annotated) type captures. For example, \kwat{\{File\}}
captures every operation on \kwat{File}, so $\fx{\{\kwa{File}\}} = \kwa{\{File.*\}}$.
A first (but not yet correct) definition of this is given in Figure \ref{fig:fx_defn}.
We then add the premise $\kwa{effects}(\hat \tau) \subseteq \varepsilon_s$,
which restricts imported capabilities to only those with effects selected in
$\varepsilon_s$. The updated rule for typing \kwat{import} is given in Figure
\ref{fig:import_rule_2}.

\begin{figure}
\vspace*{-4mm}
\begin{itemize}
	\setlength\itemsep{-0.2em}
\item[] $\kwa{effects} :: \hat \tau \rightarrow \varepsilon$
	\item[] $\fx{\{ \bar r \}} = \{ r.\pi \mid r \in \bar r, \pi \in \Pi \}$
	\item[] $\fx{\hat \tau_1 \rightarrow_{\varepsilon} \hat \tau_2} = \fx{\hat \tau_1} \cup \varepsilon \cup \fx{\hat \tau_2}$
\end{itemize}
\vspace*{-5mm}
\caption{A first (incorrect) definition of $\kwa{effects}$.}
\label{fig:fx_defn}
\end{figure}

\begin{figure}
\vspace*{-2mm}
\[
\begin{array}{c}
\infer[\textsc{($\varepsilon$-Import2-Bad)}]
	{ \hat \Gamma \vdash \import{\varepsilon_s}{x}{\hat e}{e}: \kwa{annot}(\tau, \varepsilon_s)~\kw{with} \varepsilon \cup \varepsilon_1 }
	{ \hat \Gamma \vdash \hat e: \hat \tau~\kw{with} \varepsilon_1 & x: \kwa{erase}(\hat \tau) \vdash e: \tau & \kwa{effects}(\hat \tau) \subseteq \varepsilon_s}

\end{array}
\]
\vspace*{-5mm}
\caption{A second (still incorrect) rule for type-and-effect checking $\kwa{import}$ expressions.}
\vspace*{-5mm}
\label{fig:import_rule_2}
\end{figure}

There are still issues with this second rule, as the annotations on one import
can be broken by another import. To illustrate, consider Figure
\ref{fig:rule_import2_counterexample}
where two\footnote{Our formalisation only permits a single capability
  to be imported, but this discussion leads to a generalisation needed
  for the rules to be safe when multiple capabilities can be imported.
  In any case, importing multiple capabilities can be handled with an
  encoding of pairs.} capabilities are imported. This program imports
a function $\kwa{go}$ which, when given a
$\Unit \rightarrow_{\varnothing} \Unit$ function with no effects, will
execute it. The other import is $\kwa{File}$. The unannotated code
creates a $\Unit \rightarrow \Unit$ function which writes to
$\kwa{File}$ and passes it to $\kwa{go}$, which subsequently incurs
$\kwa{File.write}$.

\begin{figure}[h]
\vspace*{-5mm}
\begin{lstlisting}
import({File.*})
   go = $\lambda$x: Unit $\rightarrow_{\varnothing}$ Unit. x unit
   f = File
in
   go ($\lambda$y: Unit. f.write)
\end{lstlisting}
\vspace*{-5mm}
\caption{Permitting multiple imports will break \textsc{$\varepsilon$-Import2}.}
\vspace*{-5mm}
\label{fig:rule_import2_counterexample}
\end{figure}

In the world of annotated code, it is not possible to pass a
file-writing function to $\kwa{go}$, but because the judgement
$x: \erase{\hat \tau} \vdash e: \tau$ discards the annotations on
$\kwa{go}$, and since the file-writing function has type
$\unit \rightarrow \unit$, the unannotated world accepts it.
Although the unannotated code is allowed to incur this effect, since
its selected authority is \kwat{\{File.*\}}, this nonetheless violates the type signature of \kwat{go}. We want to prevent this.

If $\kwa{go}$ had the type $\Unit \rightarrow_{\{ \kwa{File.write} \}} \Unit$,
Figure \ref{fig:rule_import2_counterexample} would be safely rejected. However,
a modified program where a file-reading function is passed to \kwat{go} would have
the same issue. \kwat{go} is only safe when it expects every effect that the
unannotated code might pass to it. To ensure this is the case, we shall require
imported capabilities to have the authority to incur every effect in $\varepsilon_s$.
To achieve greater control in how we say this, we split the definitions of
\kwat{effects} into two separate functions, \kwat{effects} and
\kwat{ho \hyphen effects}. The latter is for higher-order effects, which are those
effects not captured directly in the function body, but rather are possible because of
what is passed into the function as an argument. If values of $\hat \tau$ possess
a capability that can be used to incur the effect $r.\pi$, then $r.\pi \in \fx{\hat \tau}$.
If values of $\hat \tau$ can incur $r.\pi$, but need to be given the capability (as a
function argument) by someone else to do so, then $r.\pi \in \hofx{\hat \tau}$.
Definitions are given in Figure \ref{fig:fx_defns}.

\begin{figure}
\vspace*{-5mm}
\begin{itemize}
	\setlength\itemsep{-0.2em}
\item[] $\kwa{effects} :: \hat \tau \rightarrow \varepsilon$
	\item[] $\fx{\{ \bar r \}} = \{ r.\pi \mid r \in \bar r, \pi \in \Pi \}$
	\item[] $\fx{\hat \tau_1 \rightarrow_{\varepsilon} \hat \tau_2} = \hofx{\hat \tau_1} \cup \varepsilon \cup \fx{\hat \tau_2}$
\end{itemize}

\begin{itemize}
	\setlength\itemsep{-0.2em}
\item[] $\kwa{ho \hyphen effects} :: \hat \tau \rightarrow \varepsilon$
	\item[] $\hofx{\{ \bar r \}} = \varnothing$
	\item[] $\hofx{\hat \tau_1 \rightarrow_{\varepsilon} \hat \tau_2} = \fx{\hat \tau_1} \cup \hofx{\hat \tau_2}$
\end{itemize}
\vspace*{-5mm}
\caption{Effect functions (corrected).}
\vspace*{-5mm}
\label{fig:fx_defns}
\end{figure}

Both $\kwa{effects}$ and $\kwa{ho \hyphen effects}$ are mutually recursive,
with base cases for resource types. Any effect can be directly
incurred by a resource on itself, hence
$\fx{\{ \bar r \}} = \{ r.\pi \mid r \in \bar r, \pi \in \Pi \}$. A
resource cannot be used to indirectly invoke some other effect, so
$\hofx{\{ \bar r \}} = \varnothing$. The mutual recursion echoes the
subtyping rule for functions: recall that functions are contravariant
in their input type and covariant in their output; likewise, both
functions recurse on the input-type using the other function, and
recurse on the output-type using the same function.

In light of these new definitions, we still require
$\fx{\hat \tau} \subseteq \varepsilon_s$ --- unannotated code must
select any effect its capabilities can incur --- but we add a new
premise $\varepsilon_s \subseteq \hofx{\hat \tau}$, which requires
any higher-order effect of the imported capabilities to be declared in
$\varepsilon_s$. Put another way, the imported capabilities must be
expecting every effect they could be given by the unannotated code
(which is at most $\varepsilon_s$). The counterexample from Figure \ref{fig:rule_import2_counterexample} is now rejected, because
$\hofx{(\Unit \rightarrow_{\varnothing} \Unit)
  \rightarrow_{\varnothing} \Unit} = \varnothing$, but
$\fx{\kwa{File}} = \{ \kwa{File.*} \} \not\subseteq \varnothing$.

This is
\textit{still} not sufficient! Consider
$\varepsilon_s \subseteq \hofx{ \hat \tau_1 \rightarrow_{\varepsilon'}
  \hat \tau_2 }$. Expanding the definition of
$\kwa{ho \hyphen effects}$, this is the same as
$\varepsilon_s \subseteq \fx{\hat \tau_1} \cup \hofx{\hat
  \tau_2}$. Let $r.\pi \in \varepsilon_s$ and suppose
$r.\pi \in \fx{\hat \tau_1}$, but $r.\pi \notin \hofx{\hat
  \tau_2}$. Then
$\varepsilon_s \subseteq \fx{\hat \tau_1} \cup \hofx{\hat \tau_2}$ is
still true, but $\hat \tau_2$ is not expecting $r.\pi$. If $\hat \tau_2$ is
a function, unannotated code could violate its annotations by passing it
a capability for $r.\pi$, even though $r.\pi$ is not a higher-order effect
of $\hat \tau_2$.

The cause of this issue is that $\subseteq$
does not distribute over $\cup$. We want a relation like
$\varepsilon_s \subseteq \fx{\hat \tau_1} \cup \hofx{\hat \tau_2}$,
which also implies $\varepsilon_s \subseteq \fx{\hat \tau_1}$ and
$\varepsilon_s \subseteq \fx{\hat \tau_2}$. Figure
\ref{fig:safe_defns} defines this: $\kwa{safe}$ is a distributive
version of $\varepsilon_s \subseteq \fx{\hat \tau}$ and
$\kwa{ho \hyphen safe}$ is a distributive version of
$\varepsilon_s \subseteq \hofx{\hat \tau}$. An amended version of
\textsc{$\varepsilon$-Import} is given in Figure \ref{fig:import_rule3},
with a new premise $\hosafe{\hat \tau}{\varepsilon_s}$, capturing the
notion that imported capabilities must be expecting the effects they could
be passed by the unannotated code (which is at most $\varepsilon_s$).

\begin{figure}
\vspace*{-5mm}
\noindent
$\fbox{$\safe{\hat \tau}{\varepsilon}$}$

\vspace{-2mm}
\[
\begin{array}{c}
\infer[\textsc{(Safe-Resource)}]
	{ \kwa{safe}(\{ \bar r \}, \varepsilon) }
	{} 
\hspace{5ex}
	
\infer[\textsc{(Safe-Arrow)}]
	{\kwa{safe}(\hat \tau_1 \rightarrow_{\varepsilon'} \hat \tau_2, \varepsilon)}
	{\varepsilon \subseteq \varepsilon' & \kwa{ho \hyphen safe}(\hat \tau_1, \varepsilon) & \kwa{safe}(\hat \tau_2, \varepsilon)} \\[3ex]

\end{array}
\]

\noindent
$\fbox{$\hosafe{\hat \tau}{\varepsilon}$}$

\vspace{-2mm}
\[
\begin{array}{c}
\infer[\textsc{(HOSafe-Resource)}]
	{ \kwa{ho \hyphen safe}( \{ \bar r \}, \varepsilon)} 
	{}
\hspace{5ex}

\infer[\textsc{(HOSafe-Arrow)}]
	{ \kwa{ho \hyphen safe}( \hat \tau_1 \rightarrow_{\varepsilon'} \hat \tau_2, \varepsilon ) }
	{ \kwa{safe}(\hat \tau_1, \varepsilon)  & \kwa{ho \hyphen safe}(\hat \tau_2, \varepsilon) }
\end{array}
\]
\vspace*{-5mm}
\caption{Safety judgements in $\epscalc$.}
\vspace*{-7mm}
\label{fig:safe_defns}
\end{figure}

\begin{figure}
\[
\begin{array}{c}
\infer[\textsc{($\varepsilon$-Import3-Bad)}]
	{ \hat \Gamma \vdash \kwa{import}(\varepsilon_s)~x = \hat e~\kw{in} e: \kwa{annot}(\tau, \varepsilon_s)~\kw{with} \varepsilon \cup \varepsilon_1 }
{{\def\arraystretch{1.4}
  \begin{array}{c}
\hat \Gamma \vdash \hat e: \hat \tau~\kw{with} \varepsilon_1
~~~~~~
\kwa{effects}(\hat \tau) \subseteq \varepsilon_s \\[-.5mm]
\hosafe{\hat \tau}{\varepsilon_s} ~~~~~~ x: \kwa{erase}(\hat \tau) \vdash e: \tau
  \end{array}}} 
\end{array}
\]
\vspace*{-5mm}
\caption{A third (still incorrect) rule for type-and-effect checking $\kwa{import}$ expressions.}
\vspace*{-5mm}
\label{fig:import_rule3}
\end{figure}

The premises so far restrict what authority can be selected by
unannotated code, but consider the example
$\hat e = \import{\varnothing}{x}{\unit}{\lambda f: { \File
  }.~\kwa{f.write}}$. The unannotated code selects no capabilities and
returns a function which takes $\kwa{File}$ and incurs
$\kwa{File.write}$. This satisfies the premises in
\textsc{$\varepsilon$-Import3}, but its type would be the pure function
$\{ \File \} \rightarrow_{\varnothing} \Unit$.

Speaking more generally, suppose the unannotated code evaluates to a
function of type $f$, which is annotated to $\annot{f}{\varepsilon_s}$.
Suppose $\annot{f}{\varepsilon_s}$ is
invoked at a later point, back in the annotated world, incurring $r.\pi$.
 What is the source of $r.\pi$? If $r.\pi$ was selected by the
$\kwa{import}$ expression surrounding $f$, it is safe for
$\annot{f}{\varepsilon_s}$ to incur this effect. Otherwise,
$\annot{f}{\varepsilon_s}$ may have been passed, as an argument,
a capability to do $r.\pi$, in which case $r.\pi$ is a higher-order effect
of $\annot{f}{\varepsilon_s}$. If the argument
is a function, then $r.\pi \in \varepsilon_s$ by the soundness of our
calculus. But if the argument is a resource literal $r$, then
$\annot{f}{\varepsilon_s}$ could exercise $r.\pi$ without declaring it
in $\varepsilon_s$ --- this we do not yet account for.

To make $\varepsilon_s$ contain every effect captured by resources
passed into $\annot{f}{\varepsilon_s}$ as arguments, we inspect $f$
for resource types. For example, if the unannotated code evaluates to
a function of type $\kwa{ \{ File \} \rightarrow \Unit}$, we need
$\kwa{ \{ File.* \} } \in \varepsilon_s$. To do this, we add a new
premise $\hofx{\annot{\tau}{\varnothing}} \subseteq
\varepsilon_s$. Because $\kwa{ho \hyphen effects}$ is only defined on
annotated types, we first annotate $\tau$ with $\varnothing$, and since we
are only inspecting the resources passed into $f$ as arguments, our choice
of annotation doesn't matter.

Now we can handle the example from before. The unannotated code
types via the judgement
$x: \Unit \vdash \lambda f: \{ \File \}.~\kwa{f.write}: \{ \File \}
\rightarrow \Unit$. Its higher-order effects are
$\hofx{\annot{ \{ \File \} \rightarrow \Unit}{\varnothing}} = \{
\kwa{File.*} \}$, but $\{ \kwa{File.*} \} \not\subseteq \varnothing$,
so the example is safely rejected.

The final version of \textsc{$\varepsilon$-Import} is given in Figure
\ref{fig:import_rule}. With it, we can now model the example from the
beginning of this section, where the $\kwa{Logger}$ selects the
$\kwa{File}$ capability and exposes an unannotated function
$\kwa{log}$ with type $\Unit \rightarrow \Unit$ and implementation
$e$. The expected least authority of $\kwa{Logger}$ is
$\{ \kwa{File.append} \}$, so its corresponding $\kwa{import}$
expression would be
$\import{\kwa{File.append}}{f}{\kwa{File}}{\lambda x: \Unit.~e}$. The
imported capability is $ f = \kwa{File}$, which has type $\{ \kwa{File} \}$, and
$\fx{\{\File\}} = \{ \kwa{File.*} \} \not\subseteq \{
\kwa{File.append} \}$, so this example safely rejects:
$\kwa{Logger.log}$ has authority to do anything with $\kwa{File}$, and
its implementation $e$ might be violating its stipulated least
authority $\{ \kwa{File.append} \}$.


\begin{figure}[h]
\vspace*{-7mm}
\[
\begin{array}{c}

\infer[\textsc{($\varepsilon$-Import)}]
	{ \hat \Gamma \vdash \kwa{import}(\varepsilon_s)~x = \hat e~\kw{in} e: \kwa{annot}(\tau, \varepsilon_s)~\kw{with} \varepsilon_s \cup \varepsilon_1 }
{{\def\arraystretch{1.4}
  \begin{array}{c}
\kwa{effects}(\hat \tau) \cup \hofx{\annot{\tau}{\varnothing}}\subseteq \varepsilon_s \\
\hat \Gamma \vdash \hat e: \hat \tau~\kw{with} \varepsilon_1 ~~~~~~ \kwa{ho \hyphen safe}(\hat \tau, \varepsilon_s) ~~~~~~ x: \kwa{erase}(\hat \tau) \vdash e: \tau
  \end{array}}} 
 
\end{array}
\]
\vspace*{-5mm}
\caption{The final rule for typing imports.}
\vspace*{-7mm}
\label{fig:import_rule}
\end{figure}

\OMIT{
\subsection{Soundness ($\epscalc$)} 

In this section we sketch a proof of soundness of $\epscalc$ by showing the usual
progress and preservation properties hold. For a more thorough,
formal discusion, we refer readers to our technical report
\cite{ecs:2018:aaron-tr}. The proof of progress is routine.
 
\begin{theorem}[$\epscalc$ Progress]
If $\hat \Gamma \vdash \hat e_A: \hat \tau_A ~\kw{with} \varepsilon_A$ and
$\hat e_A$ is not a value, then $\hat e_A \longrightarrow \hat e_B \mid \varepsilon$,
where $\hat \Gamma \vdash \hat e_B: \hat \tau_B~\kw{with} \varepsilon_B$ and
$\hat \tau_B <: \hat \tau_A$ and $\varepsilon_B \cup \varepsilon \subseteq
\varepsilon_A$, for some $\hat e_B$, $\varepsilon$, $\hat \tau_B$, $\varepsilon_B$. 
\end{theorem}

\begin{proof}
By induction on derivations of $\hat \Gamma \vdash \hat e: \hat \tau ~\kw{with}
\varepsilon$.
\end{proof}

The proof of preservation is by induction on derivations of $\hat \Gamma \vdash \hat
e_A: \hat \tau_A~\kw{with} \varepsilon_A$ and $\hat e_A \longrightarrow \hat e_B$.
This proof is also routine, except for the case where the reduction rule used is
\textsc{E-Import2}. To do this, we need the help of the next two lemmas.

Firstly, since $\varepsilon_s$ is an upper bound on what effects the unannotated code
can incur, it is also an upper bound on what effects can be incurred by the capabilities
passed into the unannotated code. Therefore if we reannotate the type of the imported
capability with $\varepsilon_s$, we should get a more general function type
$\annot{\erase{\hat \tau}}{\varepsilon_s}$. This result is given as the pair of lemmas
 below.

\begin{lemma}
If $\fx{\hat \tau} \subseteq \varepsilon$ and
$\hosafe{\hat \tau}{\varepsilon}$, then $\hat \tau <: \annot{\erase{\hat \tau}}
{\varepsilon}$.
\end{lemma}

\begin{lemma}
If $\hofx{\hat \tau} \subseteq \varepsilon$ and $\hosafe{\hat \tau}{\varepsilon}$,
then $\hat \tau <: \annot{\erase{\hat \tau}}{\varepsilon}$.
\end{lemma}

\begin{proof}
By simultaneous induction on derivations of $\hosafe{\hat \tau}{\varepsilon}$ and
$\safe{\hat \tau}{\varepsilon}$.
\end{proof}

As function types are contravariant in their input, the subtyping and subsetting
relations on input types also flip. This is why there are two lemmas, one for each
direction.

Now, in the case where rule \textsc{E-Import2} is applied, the reduction has the form
$\import{\varepsilon_s}{x}{\hat v}{e} \longrightarrow
[\hat v/x]\annot{e}{\varepsilon_s} \mid \varnothing$. Since
$x: \erase{\hat \tau} \vdash e: \tau$, it is reasonable to expect that (A) $\hat
\Gamma \vdash \annot{\hat \tau}{\varepsilon_s}~\kw{with} \varepsilon_s$ is true
--- the reduction annotates $e$ with $\varepsilon_s$, so the type after annotating
should be the type of $e$, annotated with $\varepsilon_s$, i.e.
$\annot{\tau}{\varepsilon_s}$. Now if judgement (A) holds, then $\hat \Gamma
\vdash [\hat v/x]\annot{e}{\varepsilon_s}: \annot{\tau}{\varepsilon_s}~\kw{with}
\varepsilon_s$ would hold by substitution (remember that evaluation is strict, so
we only ever substitute values). That judgement (A) is true is the subject of the
following lemma.

\begin{lemma}[$\epscalc$ Annotation]
If 
(1) $\hat \Gamma \vdash \hat v: \hat \tau ~ \kw{with} \varnothing$,
(2) $\Gamma, y: \erase{\hat \tau} \vdash e: \tau$,
(3) $\fx{\hat \tau} \cup \hofx{\annot{\tau}{\varnothing}} \cup
    \fx{\annot{\Gamma}{\varnothing}} \subseteq \varepsilon_s$, and
(4) $\hosafe{\hat \tau}{\varepsilon_s}$,
then $\hat \Gamma, \annot{\Gamma}{\varepsilon_s}, y: \hat \tau \vdash
	\annot{\tau}{\varepsilon_s} ~\kw{with} \varepsilon_s$

%\begin{enumerate}
%	\item $\hat \Gamma \vdash \hat v: \hat \tau ~ \kw{with} \varnothing$
%	\item $\Gamma, y: \erase{\hat \tau} \vdash e: \tau$
%	\item $\fx{\hat \tau} \cup \hofx{\annot{\tau}{\varnothing}} \cup
%		\fx{\annot{\Gamma}{\varnothing}} \subseteq \varepsilon_s$
%	\item $\hosafe{\hat \tau}{\varepsilon_s}$
%\end{enumerate}
\end{lemma}

\begin{proof}
By induction on derivations of $\Gamma, y: \erase{\hat \tau_1} \vdash e: \tau$.

\textit{Case: \textsc{T-Var}}. Then $e = x$. If $x \neq y$ use
\textsc{$\varepsilon$-Var} and \textsc{$\varepsilon$-Subsume}. Otherwise $x = y$.
Then $y: \erase{\hat \tau} \vdash x: \tau$ implies that $\erase{\hat \tau} = \tau$.
Apply the approximation lemma and simplify to obtain $\hat \tau <:
\annot{\tau}{\varepsilon_s}$, then use \textsc{$\varepsilon$-Subsume} to get the
result.

\textit{Case: \textsc{T-Resource}}. Use \textsc{$\varepsilon$-Resource} and
\textsc{$\varepsilon$-Subsume}.

\textit{Case: \textsc{T-Abs}}. Use inversion to get a judgement for the body of the
function $\Gamma, y: \erase{\hat \tau}, x: \tau_2 \vdash e_{body}: \tau_3
~\kw{with} \varepsilon_s$. Apply the inductive hypothesis to $e_{body}$ with
$\Gamma, x: \tau_2$ as the context in which $e_{body}$ typechecks, noting the
premises for the inductive application are satisfied because
$\hofx{\annot{\tau}{\varepsilon_s}} \subseteq \varepsilon_s$ implies
$\fx{\annot{\hat \tau_1}{\varnothing}}$. Then use \textsc{$\varepsilon$-Abs} and
\textsc{$\varepsilon$-Subsume}.

\textit{Case: \textsc{T-App}}. Apply the inductive assumption to the subexpressions,
then use \textsc{$\varepsilon$-App} and simplify.

\textit{Case: \textsc{T-OperCall}}. Apply the inductive hypothesis to the receiver and
use \textsc{$\varepsilon$-OperCall}. This gives the approximate effects $\varepsilon_s
\cup \{ \overline{r}.\pi \}$. Consider where the binding for $\{ \bar r \}$ is in
$\hat \Gamma, \annot{\Gamma}{\varepsilon_s}, y: \hat \tau$ and conclude that
$\{ \bar r.\pi \} \subseteq \varepsilon_s$.
\end{proof}

Armed with the annotation and approximation lemmas, we can prove preservation.

\begin{theorem}[$\epscalc$ Preservation]
If $\hat \Gamma \vdash \hat e_A: \hat \tau_A~\kw{with} \varepsilon_A$ and
$\hat e_A \longrightarrow \hat e_B \mid \varepsilon$, then $\hat \Gamma \vdash
\hat e_B: \hat \tau_B~\kw{with} \varepsilon_B$, where $\hat e_B <: \hat e_A$
and $\varepsilon \cup \varepsilon_B \subseteq \varepsilon_A$, for some $\hat e_B$,
$\varepsilon$, $\hat \tau_B$, $\varepsilon_B$.
\end{theorem}

\begin{proof}
By induction on derivations of $\hat \Gamma \vdash \hat e_A: \hat \tau_A~\kw{with}
\varepsilon_A$ and $\hat e_A \longrightarrow \hat e_B \mid \varepsilon$. We shall
prove the case where the rule used is \textsc{$\varepsilon$-Import}. Then $e_A =
\import{\varepsilon_s}{x}{\hat e}{e}$. If the reduction rule used was
\textsc{E-Import1} then the result follows by applying the inductive hypothesis to
$\hat e$. Otherwise $\hat e$ is a value and the reduction used was
\textsc{E-Import2}. Apply the annotation lemma with $\Gamma = \varnothing$ to
get the judgement $\hat \Gamma, x: \hat \tau \vdash \annot{e}{\varepsilon_s}~
\kw{with} \varepsilon_s$. By assumption, $\hat \Gamma \vdash \hat v: \hat \tau~
 \kw{with} \varnothing$, so the substitution lemma applies, giving $\hat \Gamma
\vdash [\hat v/x]\annot{e}{\varepsilon}: \annot{\tau}{\varepsilon_s}$. Then
$\varepsilon_B = \varepsilon_s = \varepsilon_A \cup \varepsilon$ and $\tau_A =
\tau_B = \annot{\tau}{\varepsilon_s}$.

\end{proof}

With progress and preservation, we can prove the soundness theorem.

\begin{theorem}[$\epscalc$ Single-step Soundness]
If $\hat \Gamma \vdash \hat e_A: \hat \tau_A~\kw{with} \varepsilon_A$ and $\hat e_A$ is not a value, then $\hat e_A \longrightarrow \hat e_B~|~\varepsilon$, where $\hat \Gamma \vdash \hat e_B: \hat \tau_B~\kw{with} \varepsilon_B$ and $\hat \tau_B <: \hat \tau_A$ and $\varepsilon_B \cup \varepsilon \subseteq \varepsilon_A$, for some $\hat e_B, \varepsilon, \hat \tau_B, \varepsilon_B$.
\end{theorem}
}

\vspace{-0.5cm}
\section{Applications}
\vspace{-0.3cm}
\label{s:app}

In this section, we examine a number of scenarios to show how capabilities can help
developers reason about the effects and behaviour of code. In each story we will
discuss some Wyvern code before translating it to $\epscalc$ and explaining how its
rules apply. In doing this, we hope to convince the reader of the benefits of
capability-based reasoning, and that $\epscalc$ captures the intuitive properties of
capability-safe languages like Wyvern.

\vspace{-0.5cm}
\subsection{Unannotated Client}
\vspace{-0.2cm}

A \kwat{logger} module, when given \kwat{File}, exposes a \kwat{log} function
which incurs the \kwat{File.append} effect. The \kwat{client} module, possessing the
\kwat{logger} module, exposes an unannotated function \kwat{run}. While
\kwat{logger} has been annotated, \kwat{client} has not. If \kwat{client.run} is
executed, what effects might it have? Code for this example is given below. 

\begin{lstlisting}
module def logger(f: {File}):Logger

def log(): Unit with {File.append} =
    f.append(``message logged'')
\end{lstlisting}

\begin{lstlisting}
module def client(logger: Logger)

def run(): Unit =
   logger.log()
\end{lstlisting}

\begin{lstlisting}
require File

instantiate logger(File)
instantiate client(logger)

client.run()
\end{lstlisting}

A translation into $\epscalc$ is given below. Lines 1-3 and 5-8 define
\kwat{MakeLogger} and \kwat{MakeClient}, which instantiate the \kwat{logger} and
\kwat{client} modules respectively (represented as functions). Lines 10-14 define
\kwat{MakeMain}, which returns a function which, when executed, instantiates all
other modules and invokes the code in the body of \kwat{main}. Program execution
begins on line 16, where \kwat{main} is given the initial capabilities (just \kwat{File}
in this case).

\begin{lstlisting}
let MakeLogger =
   ($\lambda$f: File.
      $\lambda$x: Unit. f.append) in
          
let MakeClient =
   ($\lambda$logger: Unit $\rightarrow_{ \{ \kwa{File.append} \}}$ Unit.
      import(File.append) l = logger in
         $\lambda$x: Unit. l unit) in
          
let MakeMain =
   ($\lambda$f: File.
         let loggerModule = MakeLogger f in
         let clientModule = MakeClient loggerModule in
         clientModule unit) in

MakeMain File
\end{lstlisting}

The interesting part  is on line $7$, where the unannotated code selects $\{ \kwa{File.append} \}$ as its authority. This matches the effects of \kwat{logger}, i.e.
 $\kwa{effects}(\Unit \rightarrow_{\{\kwa{File.append}\}} \Unit) = \{
 \kwa{File.append} \}$. The unannotated code typechecks by \textsc{$\varepsilon$-Import}, approximating its effects as $\kwa{\{ \kwa{File.append} \}}$.

\vspace{-0.5cm}
\subsection{Unannotated Library}
\vspace{-0.2cm}

The next example inverts the roles of the last scenario. Now, the annotated 
\kwat{client} wants to use the unannotated \kwat{logger}, which captures 
\kwat{File} and exposes a single function \kwat{log}, which incurs the
 \kwat{File.append} effect. The implementation of \kwat{client.run} executes
 \kwat{logger.log}; it is annotated with $\varnothing$, so this violates its interface.

\begin{lstlisting}
module def logger(f: {File}): Logger

def log(): Unit =
    f.append(``message logged'')
\end{lstlisting}

\begin{lstlisting}
module def client(logger: Logger)

def run(): Unit with {File.append} =
   logger.log()
\end{lstlisting}

\begin{lstlisting}
require File

instantiate logger(File)
instantiate client(logger)

client.run()
\end{lstlisting}

The translation is given below. On lines 3-4, the unannotated code is wrapped in an $\kwa{import}$ expression selecting $\{ \kwa{File.append} \}$ as its authority. The implementation of $\kwa{logger}$ actually abides by this, but since it captures
\kwat{File} it could, in general, do anything to \kwat{File}; therefore,
\textsc{$\varepsilon$-Import} rejects this example. Formally, the imported capability
has the type \kwat{ \{ File \} }, but $\fx{\{ \File \}} = \{ \kwa{File}.* \}
\not\subseteq \{ \kwa{ File.append } \}$.

\begin{lstlisting}
let MakeLogger =
   ($\lambda$f: File.
      import(File.append) f = f in
         $\lambda$x: Unit. f.append) in

let MakeClient =
	($\lambda$logger: Logger.
	   $\lambda$x: Unit. logger unit) in

let MakeMain =
   ($\lambda$f: File.
      let loggerModule = MakeLogger f in
      let clientModule = MakeClient loggerModule in
      clientModule unit) in

MakeMain File
\end{lstlisting}

The only way for this to typecheck would be to annotate $\kwa{client.run}$ as having every effect on $\File$. This demonstrates how the effect-system of $\epscalc$ approximates unannotated code: it simply considers it as having every effect which could be incurred on those resources in scope, which here is $\kwa{File}.*$.

\vspace{-0.5cm}
\subsection{Higher-Order Effects}
\vspace{-0.2cm}

Here, $\kwa{Main}$ gains its functionality from a plugin. Plugins might be written by
third-parties, so we may not be able to view their source code, but still want to reason
about the authority they exercise. In this example, \kwat{plugin} has access to
\kwat{File}, but its interface does not permit it to perform any operations on
\kwat{File}. It tries to subvert this by wrapping \kwat{File} inside a function and
passing it to \kwat{malicious}, which invokes \kwat{File.read} in a higher-order
manner in an unannotated context.

\begin{lstlisting}
module malicious

def log(f: Unit $\rightarrow$ Unit): Unit
   f()
\end{lstlisting}

\begin{lstlisting}
module plugin
import malicious

def run(f: {File}): Unit with $\varnothing$
   malicious.log($\lambda$x:Unit. f.read)
\end{lstlisting}

\begin{lstlisting}
require File
import plugin

plugin.run(File)
\end{lstlisting}

This example shows how higher-order effects can obfuscate potential security risks. On line 3 of $\kwa{malicious}$, the argument to $\kwa{log}$ has type $\Unit \rightarrow \Unit$. The body of $\kwa{log}$ types with the \textsc{T-}rules, which do not approximate effects. It is not clear from inspecting the unannotated code that a $\kwa{File.read}$ will be incurred. To realise this requires one to examine the source code of both $\kwa{plugin}$ and $\kwa{malicious}$.

A translation is given below. On lines 2-3, the $\kwa{malicious}$ code selects its authority as $\varnothing$, to be consistent with the annotation on $\kwa{plugin.run}$. \textsc{$\varepsilon$-Import} safely rejects this: when the unannotated
 code is annotated with $\varnothing$, it has type $\{ \File \} \rightarrow_{\varnothing} \Unit$, but the higher-order effects of this type are
\kwat{ \{ File.* \} }, which are not contained in the selected authority $\varnothing$.

\begin{lstlisting}
let malicious =
   (import($\varnothing$) y=unit in
      $\lambda$f: Unit $\rightarrow$ Unit. f()) in

let plugin =
   ($\lambda$f: {File}.
      malicious($\lambda$x:Unit. f.read)) in

let MakeMain =
   ($\lambda$f: {File}.
      plugin f) in

MakeMain File
\end{lstlisting}

To get this example to typecheck, the program would have to be rewritten to explicitly
say that plugins can exercise arbitrary authority over \kwat{File}, by changing the
selected authority of \kwat{import} and the annotation on \kwat{plugin.run}.

\vspace{-0.5cm}
\subsection{Resource Leak}
\vspace{-0.2cm}

This is another example which obfuscates an unsafe effect by invoking it in a higher-order manner. The setup is the same, except the function which $\kwa{plugin}$ passes to $\kwa{malicious}$ now returns $\kwa{File}$ when invoked. $\kwa{malicious}$ uses this function to obtain $\kwa{File}$ and directly invokes $\kwa{read}$ upon it, violating the declared purity of $\kwa{plugin}$.

\begin{lstlisting}
module malicious

def log(f: Unit $\rightarrow$ File):Unit
   f().read
\end{lstlisting}

\begin{lstlisting}
module plugin
import malicious

def run(f: {File}): Unit with $\varnothing$
   malicious.log($\lambda$x:Unit. f)
\end{lstlisting}

\begin{lstlisting}
require File

import plugin

plugin.run(File)
\end{lstlisting}

The translation is given below. The unannotated code in $\kwa{malicious}$ is on lines
5-6. It has selected authority is $\varnothing$, to be consistent with the annotation on
$\kwa{plugin}$. Nothing is being imported, so the $\kwa{import}$ binds $\kwa{y}$ to
 $\unit$. This example is rejected by \textsc{$\varepsilon$-Import} because the
  premise $\varepsilon = \fx{\hat \tau} \cup \hofx{\annot{\tau}{\varepsilon}}$ is not satisfied. In this case, $\varepsilon = \varnothing$ and $\tau = \kwa{ (Unit \rightarrow
\{ File \}) \rightarrow Unit }$. Then $\annot{\tau}{\varepsilon} = \kwa{ (Unit
\rightarrow_{\varnothing} \{ File \}) \rightarrow_{\varnothing} Unit }$ and
$\hofx{\annot{\tau}{\varepsilon}} = \{ \kwa{File.*} \}$. Thus, the premise cannot
be satisfied and the example is safely rejected.


\begin{lstlisting}
let malicious =
   (import($\varnothing$) y=unit in
      $\lambda$f: Unit $\rightarrow$ {File}. f().read) in

let plugin =
   ($\lambda$f: {File}.
      malicious($\lambda$x:Unit. f)) in

let MakeMain =
   ($\lambda$f: {File}.
      plugin f) in

MakeMain File
\end{lstlisting}
\vspace{-0.6cm}
\section{Conclusions}
\vspace{-0.4cm}

We introduced $\epscalc$, a lambda calculus with a simple notion of resources and
their operations, which allows unannotated code to be nested inside annotated code
with a new \kwat{import} construct. Its capability-safe design enables us to safely
reason about the effects of unannotated code by inspecting what capabilities are
passed into it by its annotated surroundings. Such an approach allows code to be
incrementally annotated, giving developers a balance between safety and convenience,
alleviating the verbosity that has discouraged widespread adoption of effect systems
\cite{rytz12}.

More broadly, our results demonstrate that the most basic form of capability-based reasoning---that you can infer what code can do based on what capabilities are passed to it---is not only useful for informal reasoning, but can improve formal reasoning about code by reducing the necessary annotation overhead.

\vspace{-0.6cm}
\subsection{Related Work}
\vspace{-0.4cm}

While much related work has already been discussed as part of the presentation, here we cover some additional strands related to capabilities and effects.

Capabilities were introduced by \cite{dennis66} to control which processes in an operating system had permission to access which operating system resources.
These ideas were adapted to the programming language setting, particularly by
Mark Miller \cite{miller06}, whose object-capability model constraints how permissions
may proliferate among objects in a distributed system. \cite{maffeis10} formalised
the notion of a capability-safe language and showed that a subset of Caja (a
Javascript implementation) is capability-safe. Miller's object-capability model has been
applied to more heavyweight systems, such as \cite{drossopoulou07}, which
formalises the notion of trust in a Hoare logic. Capability-safety parallels have been
explored in the operating systems literature, where similar restrictions on dynamic
 loading and resource access \cite{hunt07} enable static, lightweight analyses to
  enforce privilege separation \cite{madhavapeddy13}.

The original effect system by \cite{lucassen88} was used to determine what expressions could safely execute in parallel. Subsequent applications include determining what functions a program might invoke \cite{tang94} and what regions in memory might be accessed or updated during execution \cite{talpin94}. In these systems, ``effects'' are performed upon ``regions''; in ours, ``operations'' are performed upon ``resources'''. $\epscalc$ also distinguishes between unannotated and annotated code: only the latter will type-and-effect-check. Another capability-based effect system is the one by \cite{devriese16}, who use effect polymorphism and possible world semantics to express behavioural invariants on data structures. $\epscalc$ is not as expressive, since it only topographically inspects how capabilities can be passed around a program, but the resulting formalism and theory is much more
lightweight. Ongoing work with the Wyvern programming language includes an
effect system which partially builds on ideas from this paper \cite{melicher18}.

\vspace{-0.6cm}
\subsection{Future Work}
\vspace{-0.2cm}

Our effects only model capabilities which manipulate system resources. This
definition could be generalised to track other sorts of effects, such as stateful
updates. In our formalism, resources and their operations are fixed throughout
runtime, but we could imagine them being created and destroyed at runtime. 
Finally, other future work could incorporate polymorphic types and effects.



%% Bibliography
%\vspace{-0.5cm}
%% \bibliography{biblio}


\begin{thebibliography}{10}
\vspace{-0.2cm}
\providecommand{\url}[1]{\texttt{#1}}
\providecommand{\urlprefix}{URL }
\providecommand{\doi}[1]{https://doi.org/#1}

\bibitem{coker15}
Coker, Z., Maass, M., Ding, T., Le~Goues, C., Sunshine, J.: Evaluating the
  flexibility of the {J}ava sandbox. In: Proceedings of the 31st Annual
  Computer Security Applications Conference. pp. 1--10. ACSAC 2015, ACM, New
  York, NY, USA (2015). \doi{10.1145/2818000.2818003},
  \url{http://doi.acm.org/10.1145/2818000.2818003}

\bibitem{ecs:2018:aaron-tr}
Craig, A., Potanin, A., Groves, L., Aldrich, J.: Capabilities: Effects for
  free. Tech. rep., School of Engineering and Computer Science, Victoria
  University of Wellington, Wellington, New Zealand (2018),
  \url{https://ecs.victoria.ac.nz/Main/TechnicalReportSeries}

\bibitem{dennis66}
Dennis, J.B., Van~Horn, E.C.: Programming semantics for multiprogrammed
  computations. Commun. ACM  \textbf{9}(3),  143--155 (Mar 1966).
  \doi{10.1145/365230.365252}, \url{http://doi.acm.org/10.1145/365230.365252}

\bibitem{devriese16}
Devriese, D., Birkedal, L., Piessens, F.: Reasoning about object capabilities
  with logical relations and effect parametricity. In: IEEE European Symposium
  on Security and Privacy (2016)

\bibitem{dimoulas14}
Dimoulas, C., Moore, S., Askarov, A., Chong, S.: Declarative policies for
  capability control. In: Computer Security Foundations Symposium (2014)

\bibitem{drossopoulou07}
Drossopoulou, S., Noble, J., Miller, M.S., Murray, T.: Reasoning about risk and
  trust in an open world. In: European Conference on Object-Oriented
  Programming. pp. 451--475 (2007)

\bibitem{hunt07}
Hunt, G., Aiken, M., F\"{a}hndrich, M., Hawblitzel, C., Hodson, O., Larus, J.,
  Levi, S., Steensgaard, B., Tarditi, D., Wobber, T.: Sealing os processes to
  improve dependability and safety. SIGOPS Oper. Syst. Rev.  \textbf{41}(3),
  341--354 (Mar 2007). \doi{10.1145/1272998.1273032}

\bibitem{Kiniry2006}
Kiniry, J.R.: Exceptions in Java and Eiffel: Two Extremes in Exception Design
  and Application, pp. 288--300. Springer Berlin Heidelberg, Berlin, Heidelberg
  (2006)

\bibitem{koka14}
Leijen, D.: Koka: Programming with row polymorphic effect types. In:
  Mathematically Structured Functional Programming 2014. EPTCS (March 2014),
  \url{https://www.microsoft.com/en-us/research/publication/koka-programming-with-row-polymorphic-effect-types-2/}

\bibitem{lucassen88}
Lucassen, J.M., Gifford, D.K.: Polymorphic effect systems. In: Proceedings of
  the 15th ACM SIGPLAN-SIGACT Symposium on Principles of Programming Languages.
  pp. 47--57. POPL '88, ACM, New York, NY, USA (1988).
  \doi{10.1145/73560.73564}

\bibitem{maass16}
Maass, M.: {A Theory and Tools for Applying Sandboxes Effectively}. Ph.D.
  thesis, Carnegie Mellon University (2016)

\bibitem{madhavapeddy13}
Madhavapeddy, A., Mortier, R., Rotsos, C., Scott, D., Singh, B., Gazagnaire,
  T., Smith, S., Hand, S., Crowcroft, J.: Unikernels: Library operating systems
  for the cloud. SIGPLAN Not.  \textbf{48}(4),  461--472 (Mar 2013).
  \doi{10.1145/2499368.2451167}

\bibitem{maffeis10}
Maffeis, S., Mitchell, J.C., Taly, A.: Object capabilities and isolation of
  untrusted web applications. In: Proceedings of the 2010 IEEE Symposium on
  Security and Privacy. pp. 125--140. SP '10, IEEE Computer Society,
  Washington, DC, USA (2010). \doi{10.1109/SP.2010.16}

\bibitem{melicher17}
Melicher, D., Shi, Y., Potanin, A., Aldrich, J.: {A Capability-Based Module
  System for Authority Control}. In: M{\"u}ller, P. (ed.) 31st European
  Conference on Object-Oriented Programming (ECOOP 2017). LIPIcs, vol.~74, pp. 20:1--20:27. \doi{10.4230/LIPIcs.ECOOP.2017.20}
  
\bibitem{melicher18effects}
Melicher, D., Shi, Y., Zhao, V., Potanin, A., Aldrich, J.: Using object
  capabilities and effects to build an authority-safe module system: poster.
  In: Proceedings of the 5th Annual Symposium and Bootcamp on Hot Topics in the
  Science of Security, HoTSoS 2018. p.~29:1 (2018). \doi{10.1145/3190619.3191691}
  
\bibitem{miller03}
Miller, M., Yee, K.P., Shapiro, J.: Capability myths demolished. Tech. rep.
  (2003)

\bibitem{miller06}
Miller, M.S.: Robust Composition: Towards a Unified Approach to Access Control
  and Concurrency Control. Ph.D. thesis, Johns Hopkins University (2006)

\bibitem{nielson99}
Nielson, F., Nelson, H.R.: {Type and Effect Systems}. pp. 114--136 (1999).
  \doi{10.1145/361604.361612}

\bibitem{rytz12}
Rytz, L., Odersky, M., Haller, P.: Lightweight polymorphic effects. In:
  Proceedings of the 26th European Conference on Object-Oriented Programming.
  pp. 258--282. ECOOP (2012)

\bibitem{talpin94}
Talpin, J.P., Jouvelot, P.: The type and effect discipline. Information and
  Computation  \textbf{111}(2),  245--296 (1994)

\bibitem{tang94}
Tang, Y.M.: Control-Flow Analysis by Effect Systems and Abstract
  Interpretation. Ph.D. thesis, Ecole des Mines de Paris (1994)

\end{thebibliography}


\end{document}