\documentclass{article}

\usepackage{listings}
\usepackage{proof}
\usepackage{amssymb}
\usepackage[margin=.9in]{geometry}
%\usepackage{amsmath}
\usepackage{mathpartir}
\usepackage{mathrsfs}
\usepackage{textcomp}

\lstdefinestyle{custom_lang}{
  xleftmargin=\parindent,
  showstringspaces=false,
  basicstyle=\ttfamily,
  keywordstyle=\bfseries
}

\lstset{emph={%  
    var, def, type, new, self%
    },emphstyle={\bfseries \tt}%
}

%\newcommand{\keyw}[1]{\texttt{\textbf{#1}}}
\newcommand{\keywadj}[1]{\mathtt{#1}}
\newcommand{\keyw}[1]{\keywadj{#1}~}

\begin{document}



\section{Virtual Machine Abstract Syntax}

\[
\begin{array}{lll}
\begin{array}{lllr}
S & ::= & d; S& statements \\
  & |   & e & \\
&&\\
d & ::= & \keyw{val} f = e & declarations \\
  & |   & \keyw{def} m(\overline{x:\tau}) : \tau = S &\\
  & |   & \keyw{type} L = \tau &\\
&&\\
e & ::= & x & expressions \\
& | & \keywadj{new}~x:\tau~\{\overline{d}\}&\\
& | & e.m(\overline{e}) &\\
& | & e.f &\\
& | & \mathscr{L} &\\
&&\\
\mathscr{L} & ::= & n & literals \\
%& | & unit &\\
%& | & string &\\
\end{array}
& ~~~~~~
&
\begin{array}{lllr}
\beta & ::= & \keyw{Unit} & \textit{base type} \\
      & |   & L \\
&&\\
\tau & ::= & \beta \{\overline{\sigma}\} & type \\
%&&\\
%T    & ::= & \tau & type defn\\
%     & |   & \keyw{extend} L \keyw{with} \{x \Rightarrow \overline{\sigma}\} \\
&&\\
%p & ::= & x & paths \\
%& | & p.f &\\
%&&\\
%s & ::= & \keyw{stateful} | ~\keyw{pure} \\
%&&\\
\sigma & ::= & \texttt{val} \; f:\tau & decl \; type\\
       & |   & \keyw{def} m(\overline{x:\tau}) : \tau &\\
%       & |   & \texttt{def} \; m:\Pi \overline{x{:}\tau} . \tau \\
       & |   & \texttt{type} \; L = \tau &\\ % will be T
%       & |   & \texttt{type} \; L &\\
&&\\
\end{array}
\end{array}
\]

Notation: overbar means a list of elements, as in Java

%Notes on semantics:

%\begin{itemize}

%\end{itemize}

\section{Standard prelude}

\begin{lstlisting}
type Int
    def +(i:Int):Int
    def -(i:Int):Int
    def *(i:Int):Int
    def /(i:Int):Int
\end{lstlisting}

\section{Virtual Machine Typing Rules}

$\fbox{$\Gamma \vdash e : \tau$}$
\[
\begin{array}{c}
\infer[\textsc{(T-Stmt)}]
  {\Gamma \vdash d; S : \tau}
  {\Gamma \vdash d : \sigma & \Gamma, \sigma \vdash S : \tau}\\[5ex]
  
\infer[\textsc{(T-Var)}]
  {\Gamma \vdash x : \tau}
  {x : \tau \in \Gamma}\\[5ex]

\infer[\textsc{(T-New)}]
	{\Gamma \vdash \keywadj{new}~x:\tau~\{\overline{d}\} : \tau}
	{\Gamma ~|~ x:\tau \vdash \overline{d} : \textit{unfold}_\Gamma(\tau)} \\[5ex]

\infer[\textsc{(T-Invk)}]
	{\Gamma \vdash e_1.m(e_2) : \tau} 
	{\Gamma \vdash e_1 : \tau_1  & \keyw{def} m(x:\tau_2'):\tau \in \textit{unfold}_\Gamma(\tau_1) & \Gamma \vdash e_2 : \tau_2 & \Gamma \vdash \tau_2 <: \tau_2'}\\[5ex]

\infer[\textsc{(T-Field)}]
	{\Gamma \vdash e.f : \tau} 
	{\Gamma \vdash e : \tau' & \keyw{val} f : \tau \in \textit{unfold}_\Gamma(\tau')}\\[5ex]

\infer[\textsc{(T-Int)}]
  {\Gamma \vdash n : \textit{Int}}
  {}\\[5ex]

\end{array}
\]

Technically $\Gamma$ is a list of $\sigma$, but we often write $x:\tau$ for $\keyw{val} x:\tau$.

$\fbox{$\Gamma ~|~ x:\tau \vdash d : \sigma$}$
\[
\begin{array}{c}
\infer[\textsc{(T-Def)}]
  {\Gamma  ~|~ x:\tau \vdash \keyw{def} m(\overline{y : \tau}) : \tau_2 = e~:~\keyw{def} m(\overline{y : \tau}) : \tau_2}
  {\Gamma, x{:}\tau \vdash \overline{\tau}~\textit{wf} & \Gamma,~x{:}\tau,~\overline{y{:}\tau} \vdash e : \tau_2' & \Gamma, x{:}\tau \vdash \tau_2' <: \tau_2}\\[5ex]

\infer[\textsc{(T-Val)}]
  {\Gamma ~|~ x:\tau \vdash \keyw{val} f = e~:~\keyw{val} f : \tau}
  {\Gamma \vdash e : \tau}\\[5ex]

\infer[\textsc{(T-Type)}]
  {\Gamma ~|~ x:\tau \vdash \keyw{type} L = \tau : \keyw{type} L = \tau}
  {\Gamma, x:\tau \vdash \keyw{type} L = \tau~\textit{wf}}\\[5ex]

\end{array}
\]

$\fbox{$\Gamma \vdash \tau_1 <: \tau_2$}$
\[
\begin{array}{c}

\infer
  {\Gamma \vdash \tau <: \tau}
  {}\\[5ex]

\infer
  {\Gamma \vdash \tau_1 <: \tau_3}
  {\Gamma \vdash \tau_1 <: \tau_2 & \Gamma \vdash \tau_2 <: \tau_3}\\[5ex]

\infer
  {\Gamma \vdash \tau <: \keyw{Unit} }
  {}\\[5ex]

\infer
  {\Gamma \vdash \tau_1 <: \tau_2}
  {\textit{canon}_\Gamma(\tau_1) = \beta_1\{\overline{\sigma_1}\} & \textit{canon}_\Gamma(\tau_2) = \beta_2\{\overline{\sigma_2}\} & \Gamma \vdash \beta_1 <: \beta_2 & \Gamma \vdash \overline{\sigma_1} <: \overline{\sigma_2}}\\[5ex]

\end{array}
\]

$\fbox{$\textit{canon}_\Gamma(\tau_1) = \beta_1\{\overline{\sigma_1}\}$}$ and $\fbox{$\Gamma \vdash \overline{\sigma_1} <: \overline{\sigma_2}$}$  and $\fbox{$\Gamma \vdash \sigma_1 <: \sigma_2$}$
\[
\begin{array}{c}

\infer
  {\textit{canon}_\Gamma(\keyw{Unit}\{\overline{\sigma}\}) = \keyw{Unit}\{\overline{\sigma}\}}
  {}\\[5ex]

\infer
  {\textit{canon}_\Gamma(L\{\overline{\sigma}\}) = L'\{\overline{\sigma'} \leftarrow \overline{\sigma}\}}
  {\keyw{type} L = \tau \in \Gamma & \textit{canon}_\Gamma(\tau) = L'\{\overline{\sigma'}\}}\\[5ex]

\infer
  {\Gamma \vdash \overline{\sigma_1} <: \overline{\sigma_2}}
  {\sigma_1' \subset \sigma_1 & \overline{\Gamma \vdash \sigma_1' <: \sigma_2}}\\[5ex]
  
\infer
  {\Gamma \vdash \texttt{val} \; f:\tau <: \texttt{val} \; f:\tau'}
  {\Gamma \vdash \tau <: \tau'}\\[5ex]

\infer
  {\Gamma \vdash \keyw{def} m(\overline{x:\tau_1}) : \tau_2 <: \keyw{def} m(\overline{x:\tau_1'}) : \tau_2'}
  {\Gamma \vdash \tau_2 <: \tau_2' & \Gamma \vdash \overline{\tau_1'} <: \overline{\tau_1}}\\[5ex]

\infer
  {\Gamma \vdash \texttt{type} \; L = \tau <: \texttt{type} \; L = \tau}
  {}\\[5ex]

\end{array}
\]

$\fbox{$\textit{unfold}_\Gamma(\tau) = \overline{\sigma}$}$
\[
\begin{array}{c}

\infer
  {\textit{unfold}_\Gamma(\keywadj{Unit}) = \bullet}
  {}\\[5ex]
  
\infer
  {\textit{unfold}_\Gamma(L) = \overline{\sigma}}
  {\keyw{type} L = \tau \in \Gamma & \textit{unfold}_\Gamma(\tau) = \overline{\sigma}}\\[5ex]
  
\infer
  {\textit{unfold}_\Gamma(\beta \{\overline{\sigma'}\}) = \overline{\sigma} \leftarrow \overline{\sigma'}}
  {\textit{unfold}_\Gamma(\beta) = \overline{\sigma}}\\[5ex]
  
\end{array}
\]

Note: $\overline{\sigma} \leftarrow \overline{\sigma'}$ means that we append the two lists, except that when the same symbol is defined in both $\overline{\sigma}$ and $\overline{\sigma'}$, we include only the (overriding) definition in $\overline{\sigma'}$.

Now, finally, type and declaration type well-formedness rules:
$\fbox{$\Gamma \vdash \tau~\textit{wf}$}$
\[
\begin{array}{c}
\infer
  {\Gamma \vdash \keyw{Unit}~\textit{wf}}
  {}\\[5ex]

\infer
  {\Gamma \vdash L~\textit{wf}}
  {\textit{unfold}_\Gamma(L) = \overline{\sigma}}\\[5ex]

\infer
  {\Gamma \vdash \beta \{\overline{\sigma'}\}~\textit{wf}}
  {\Gamma \vdash \beta ~\textit{wf} & \Gamma \vdash \overline{\sigma'}~\textit{wf}}\\[5ex]

\end{array}
\]

$\fbox{$\Gamma \vdash \sigma~\textit{wf}$}$
\[
\begin{array}{c}
\infer
  {\Gamma \vdash \keyw{def} m(\overline{y : \tau}) : \tau_2~\textit{wf}}
  {\Gamma \vdash \overline{y : \tau}~\textit{wf} & \Gamma \vdash \tau_2~\textit{wf}}\\[5ex]

\infer
  {\Gamma \vdash \keyw{val} f : \tau~\textit{wf}}
  {\Gamma \vdash \tau~\textit{wf}}\\[5ex]

\infer
  {\Gamma \vdash \keyw{type} L = \tau~\textit{wf}}
  {\Gamma, \keyw{type} L = \tau \vdash \tau~\textit{wf}}\\[5ex]


\end{array}
\]

\bibliographystyle{plain}
\bibliography{bib}

\end{document}
